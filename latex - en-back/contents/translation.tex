% !Mode:: "TeX:UTF-8"

\chapter{外文资料翻译}

\section{外文资料信息}

V. Demir 和 A. Z. Elsherbeni, 《在使用统一计算架构的图形处理器中进行时域有限差分编程》, 2010 IEEE Antennas and Propagation Society International Symposium, Toronto, ON, 2010, pp. 1-4. doi: 10.1109/APS.2010.5562117

\section{外文资料正文翻译}

\subsection{摘要}

最近图形处理器(GPU)称为执行高性能科学计算的平台。高级程序语言不能对显卡编程曾经阻碍了 GPU 的普及。最近由 NVIDIA 公司引入了统一计算架构(Compute Unified Device Architecture,CUDA)开发环境,使得 GPU 编程容易了许多。本文的工作展示了使用 CUDA 的时域有限差分(Finite-Difference Time-Domain,FDTD)的实现,展现了一个线程到元胞的映射算法和该算法的性能。

\subsection{引言}

近来 相比于 CPU,GPU 设计的发展速度相比很快。由于提供并行而带来的计算能力引起了大众对于使用 GPU 进行科学计算的关注。计算电磁学的研究者们也开始使用显卡的计算能力来进行计算,尤其是对 FDTD 的计算已经有研究报告。起初高级语言在显卡编程上很不方便。比如一些 FDTD 的实现是基于 OpenGL 的。之后有 Brook [1] 作为一个面向通用编程环境的高级语言被提出并且在 [2] 中作为 FDTD 的编程语言。更进一步,使用高级着色器语言(High Level Shader Language,HLSL)编写 FDTD 也有报导。最近 由 NVIDIA 提出的 CUDA 开发环境让 GPU 编程大幅简化。

CUDA 在 [4]-[7] 中被提出作为实现 FDTD 的编程环境。在 [4] 中,展示了使用 CUDA 的二维 FDTD,并提出了三维的 FDTD 实现方案。强调了使用级联内存的重要性和使用共享内存的高效性但没有进一步详细说明。另一个使用 CUDA 的二维 FDTD 实现方案在 [5] 中被提出,不过没有提供任何细节。一些使用 CUDA 来提升 FDTD 的效率的方法在 [6] 中提出,可以作为使用 CUDA 对 FDTD 编程的指导大纲。讨论是基于 FDTD 迭代方程的最简形式:仅考虑在计算域中的电解质导体的迭代方程,在各个$x$,$y$和$z$方向上元胞尺寸相同,因此迭代方程只包含一个迭代系数。共享内存的高校也被讨论了,但是展示的方法中每个线程块中的线程数目是有固定大小的。级联内存是 CUDA 在效率上的必须,在给出的例子中内含的满足了,但是重要性未被提及。

本文提出了一个 FDTD 实现方案。其中 FDTD 迭代方程假设了更一般的介质和不同的元胞尺寸。提出了一个线程到元胞的映射算法以及其性能。

\subsection{使用 CUDA 的 FDTD}

统一的 CUDA 中的 FDTD 公式在 [8] 中被考虑过。问题区域尺寸是$N_x\times N_y \times N_z$,其中$N_x$,$N_y$和$N_z$分别是$x$,$y$和$z$方向的上的元胞数目。如此一来,[8] 中给出的迭代磁场$x$方向的分量的方程就是
\begin{equation}
\begin{split}
H_{x}^{n+1/2}(i,j,k) =& C_{hxh}(i,j,k)H_{x}^{n-1/2}(i,j,k)\\
+ & C_{hxey}(i,j,k)(E_{y}^{n}(i,j,k+1)-E_{y}^{n}(i,j,k))\\
+ & C_{hxez}(i,j,k)(E_{z}^{n}(i,j+1,k)-E_{z}^{n}(i,j,k)).
\end{split}
\end{equation}
为了做到并行,线程被映射到元胞上去更新元胞。在映射中,一个线程块被构造为一个一维数列,如列表1中的第一列所示。另外在一维数列中线程被映射到如图1所示的 FDTD 域的 $x-y$ 平面中的元胞上。在 kernel 函数中,每一个线程被映射到一个元胞上,线程序号被映射到$i$和$j$上。然后,每个线程在$z$方向上在$for$循环中通过增加$k$序号的方式遍历元胞。对于每个$k$更新场值,因此整个 FDTD 域被遍历。

不幸的是在 FDTD 更新中内存读写起到支配作用。为了保证高性能所有的全局内存读写应被级联。通常,一个 FDTD 域的尺寸应该是任意数值。为了做到级联内存,FDTD 域要被通过填充$x$或者$y$方向上的元胞数目来使这个数目是16的整倍数,如图2所示。修正的 FDTD 域尺寸是$Nxx\times Nyy\times Nz$,其中$Nxx$,$Nyy$和$Nz$是分别是$x$,$y$和$z$方向的上的元胞数目。

在保证级联内存读写之后,数据$for$循环中的在$z$方向上的复用和适当使用共享内存也需要在 CUDA 代码中得到完善。代码的一部分如列表2所示。完善后的代码在一个 NVIDIA\textregistered Tesla\texttrademark C1060 计算卡上得到测试。立体 FDTD 问题域的尺寸递增,每秒处理的百万元胞数作为 CUDA 程序性能的测试。结果的分析如图3所示。可以看到代码每秒平均处理大约450百万个元胞。