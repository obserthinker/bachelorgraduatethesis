% !Mode:: "TeX:UTF-8"

\chapter{低噪声放大器的设计}

\section{低噪声放大器的设计流程及特点}

微波单片电路的设计过程基本就是CAD的设计过程,其设计的大致过程为为:确定低噪声放大器各个性能指标,对有源器件、无源器件进行分析,选择电路结构,优化初值得到大致电路,进一步优化参数使其更准确,对主要的参数进行灵敏度分析,设计并且优化版图,流片,芯片测试和研究。

对于低噪声放大器,在实际的设计过程当中,一定要注意一下几点:
\begin{enumerate}
	\item 放大器中放大管的选择。
	\item 输入输出匹配电路的设计原则。
	\item 电路中需要注意的问题:一般对于低噪声放大器采用高Q值的电感完成偏置和匹配功能,由于电阻会产生附加的热噪声,放大器的输入端尽量避免直接连接到偏置电阻上。
	\item 目前低噪声放大器方面的设计手段:LNA基本上采用ADS。
	\item 目前同行业低噪声放大器的发展水平:随着半导体器件的发展,低噪声放大器的性能不断提高,采用 PHEMT 场效应晶体管的低噪声放大器在 800MHz 频段噪声系数可达到 0.4dB,增益约为 17dB 左右,1900MHz 频段噪声系数可达到 0.6dB,增益为 15dB 左右。
\end{enumerate}

\section{仿真软件、芯片及基板和金属层的选择}

从20世纪80年代开始,微波电路技术的应用方向已经由传统的波导和同轴线元器件逐渐转移到微波平面电路系统,但现实的问题是,微波平面电路设计一直以来都是一项比较困难和复杂的工作,在实践中需要工程师不断的调试才能完成。近些年来,随着市场的需求不断的提升,射频电路应用的频率越来越高,为了满足高速率信号传输的要求,信道带宽也越来越宽,对于电路各项参数的要求也越来越高 ,对于产品的要求也越来越多样化,产品的尺寸也是要求越来越小,与此同时,产品的设计周期却越来越短,因此比较传统的方法已经不能满足对于现代电路设计的要求,应用微波仿真软件进行电路设计已经成为必然的发展趋势,美国安捷伦 (Agilent) 公司推出的大型 EDA 软件 ADS 已经成为当今微波设计的主流软件。对于本课题的设计,同样是选择采用 Agilent 公司的 ADS  (Advanced Design System) 进行设计,安捷伦 ADS 软件可以应用于航空电子、雷达、卫星通信系统的设计,也可以应用于移动通信系统,高速电路,信号完整性设计,射频和微波电路设计,天线设计,和 RX/TX 封装模块设计。本文选择 ADS 也主要有以下两个原因:(1) ADS 几乎可以支持微波器件和系统中所有类型电路的设计,不论是简单的还是复杂的电路结构,不论是离散的微波模块还是用于通信、航天等其他用途的单片电路,所以 ADS 是当今国内各理工科大学和各研究所进行微波单片电路电设计时比较常用的软件。(2) Agilent 公司和很多家具有相当规模流片实力的半导体厂商合作提供了各种模型库 Design Kit,设计者可以利用 Design Kit 及软件的仿真和优化进行系统指标的规划、设计与评估。

关于芯片的选择,对于国内来说,目前还没有商业化的毫米波芯片,供应商主要是集中在美国,如 TriQuint,Hittite,Mimixbroadband 和 Avago 等公司。本课题选用的芯片为 TriQuint 公司的所设计的一款低噪声放大器 TGA4508,频率范围为 30$\sim$42GHz,典型增益为 21dB,典型噪声系数为 2.8dB,偏置电压为 3V,工作电流为 40mA,采用了 0.15\quad 3MI\quad pHEMT工艺,芯片的外部尺寸为1.7mm $\times$ 0.8mm $\times$ 0.1mm,内部尺寸为0.067mm $\times$ 0.031mm $\times$ 0.004mm。

介质基片的选择:对基片的材料和电介质材料都需要进行适当的选择,不仅仅需要满足电性能的要求,还应该满足机械性能、加工性能、对于环境的适应能力以及较低的生产成本等要求。对于基片的材料,我们一般有如下要求:
\begin{enumerate}
	\item 要求较高的介电常数,尽量使电路小型化。
	\item 较低的损耗。
	\item 在已经给定的频率和温度范围内,介电常数稳定。
	\item 表面光洁度高、击穿强度高。
	\item 导热性好,适用与较大的功率。
	\item 适应环境的能力较强。
\end{enumerate}

当工作的频率提高到毫米波段时,主要的矛盾并不是减小尺寸,恰恰相反,尺寸如果太小的话反而会影响精度;在这种情况下,由于导体趋肤深度的减小,更加要求基片的表面光洁度要高,在这种情况下,更多的采用介电常数比较低,但是表面光洁度比较高的石英。

对于金属材料的使用,有如下的要求:
\begin{enumerate}
	\item 高的电导率
	\item 较低的电阻温度系数。
	\item 较低的电阻温度系数。
	\item 较好的刻蚀性和可焊接。
	\item 较好的刻蚀性和可焊接。
\end{enumerate}

较好的刻蚀性和可焊接根据各种金属的数据我们可以看出,导电性较好的金属,例如金、银、铜等附着性能很差;与之相反,导电性能很差的钼、铬、钽等的附着性能却很好。因此,在基片上淀积导电性能良好的导电性良好的金属 之前,可以先蒸一薄层的铬、钼等金属作为媒介,然后再把导电良好的金属附着于媒介金属上,此时媒介金属虽然导电率差,但因其蒸发厚度只有几十到几百埃的数量级,比其趋肤深度要小得多,因此电流的分布,完全可以穿透此薄层,而主要分布在导电良好的主金属上,因此对微带线损耗的影响极微。基于以上综合的原因,本课题的基片选用毫米波段常用的 Rogers\quad RT/Duroid 5880 板材,基片的厚度为 0.254mm,介电常数为 2.2。金属层采用了 1oz 的铜(厚度约为 0.035 mm)。

\section{芯片资料手册分析}

放大器在设计之前必须要详细了解晶体管的详细参数和各方面的性能,这是选择晶体管并进行设计的基础。一般情况下半导体公司都会提供晶体管的芯片资料手册(Datasheet),是设计者获得关于晶体管资料的重要途径。对于本课题所选用的TGA4508,资料手册上也给出了很多重要的信息。TGA4508是一款Ka波段的低噪声放大器,典型频率范围为30~42GHz,典型增益为21dB,典型噪声系数为2.8dB,下图\ref{pics/3-1}是手册中给出的增益和回波损耗曲线。
\pic[htbp]{增益和回波损耗曲线}{width=0.7\linewidth}{pics/3-1}

在25 \textcelsius 的条件下,电特性参数,如图\ref{pics/3-2}。
\pic[htbp]{25\textcelsius 的条件下电特性参数}{width=0.7\linewidth}{pics/3-2}

在偏置电压为3V,偏置电流分别为 40mA 和 60mA 的情况下,资料手册所给出的输入回波损耗,如下图\ref{pics/3-3}。
\pic[htbp]{输入回波损耗}{width=0.7\linewidth}{pics/3-3}

输出回波损耗,如\ref{pics/3-4}。
\pic[htbp]{输出回波损耗}{width=0.7\linewidth}{pics/3-4}

增益如图\ref{pics/3-5}。
\pic[htbp]{增益随频率变化图}{width=0.7\linewidth}{pics/3-5}

噪声系数如图\ref{pics/3-6}。
\pic[htbp]{60mA 和 40mA 下噪声系数}{width=0.7\linewidth}{pics/3-6}

根据资料手册中给出的数据,我们可以知道低噪声放大器的直流工作点,Vds =3V,Ids=40mA。这就完成了低噪声放大器设计最基础的第一部分。然后根据如上的指标,我们就可以开始进行低噪声放大器的设计了。

\section{晶体管直流工作点的扫描}

在设计之前,要对晶体管进行直流工作点的扫描,具体的操作如下:第一步先在ADS中建立一个新的工程:(1)运行 ADS2011,选择 File$\rightarrow$ New$\rightarrow$ Workspace 命令,弹出“New Workspace Wizard”(新建工程)对话框,可以看见对话框中已经存在了默认的工作路径(可以手动进行改变)。如下图\ref{pics/3-7}所示:
\pic[htbp]{New Workspace Wizard}{width=0.7\linewidth}{pics/3-7}

点击Next按钮,然后如下图\ref{pics/3-8}:
\pic[htbp]{New Workspace Wizard Add Libraries}{width=0.7\linewidth}{pics/3-8}

再次Next,如图\ref{pics/3-9}:
\pic[htbp]{New Workspace Wizard Library Names}{width=0.7\linewidth}{pics/3-9}

然后在选择第一项,点击Next,如图\ref{pics/3-10}:
\pic[htbp]{New Workspace Wizard Library Names}{width=0.7\linewidth}{pics/3-10}

然后 Finish 即生成新的工程。下一步是在新建的工程中,在新建一个原理图。选择 File$\rightarrow$New$\rightarrow$ Schematic命令,弹出如图\ref{pics/3-11}对话框:
\pic[htbp]{New Schematic}{width=0.7\linewidth}{pics/3-11}

点击 OK,弹出如图\ref{pics/3-12}所示,
\pic[htbp]{Schematic Wizard}{width=0.7\linewidth}{pics/3-12}

直接点击 Cancel,然后生成新的原理图。然后下一步是在原理图中进行操作,在 Schematic Design Temples栏中选择“FET$\_$curve$\_$tracer”,这是一个专门用来扫描 BJT 工作点的模板。单击 OK,此时新的原理图窗口被打开,窗口中已经出现一个专门用于对FET进行直流工作点扫描的模板,会有系统预先设好的组件和控件。对FET进行工作点扫描的过程就是一个直流仿真的过程,因此模板中的仿真控制器为直流仿真控制器,而扫描的变量是 BJT 的 CE 极电压 VCE 和 B 极电流 IBB,如图\ref{pics/3-13}。
\pic[htbp]{工作点扫描电路}{width=0.7\linewidth}{pics/3-13}

这样对晶体管进行直流工作点扫描的电路就完成了,单击工具栏中的 Simulate 执行仿真,并等待仿真结束。仿真结束后,系统弹出数据显示窗口,由于使用的是仿真模板,需要的仿真结果已经出现在窗口中,图中就是 BJT 的直流工作点扫描曲线以及 BJT 的直流工作点和功耗,如图\ref{pics/3-14}所示。
\pic[htbp]{工作点扫描电路结果}{width=0.7\linewidth}{pics/3-14}

\section{参数的扫描}

S 参数(S-Parameter)也叫散射参量,在微波频段中,使用电压和电流表示的各种参量在实际的应用中并不是特别的方便,第一个原因是因为缺少恒定的微波电压源和微波电流源;第二个原因是理想的短路或开路终端是不容易得到的,在电路的实际应用当中,器件并不是处于短路或者开路的状态,不连续的终端负载可能会导致有害的电压、电流波的反射,并且产生可能会造成器件损坏的振荡。因此这些参量很难被正确地测量。在微波技术当中,不管电路的形势是如何变化的,我们可以设法使信号源的输出功率保持不变,找到匹配的终端负载。这样根据参考面上归一化的入射波电压和归一化的反射波电压之间的关系导出的散射参量和传输参量就会显得更加方便一些。而且通过使用精确的信号源和精确的终端负载,就可以大大提升网络参数的测量精度。同时,利用 S-parameter 可以在避开不现实的终端条件以及避免造成被测器件(DUT)损坏的条件下,用二端口网络的分析方法可以确定几乎所有微波器件的特征。S-parameter 在微波电路设计、微波测量、微波系统技术规范中得到了非常广泛的应用。如下图\ref{pics/3-15}为二端口网络及其 S 参量:
\pic[htbp]{二端口网络示意图}{width=0.7\linewidth}{pics/3-15}

上图中描述了 S-parameter 的定义原理。图中的 $a_1$ 和 $b_1$ 分别是端口 1 的归一化入射电压波和反射电压波;而 $a_2$ 和 $b_2$ 分别为端口2的归一化入射电压波和反射电压波;输入参考面和输出参考面分别为 $T_1$ 和 $T_2$。利用两个参考面上的入射波电压来表示两个参考面的反射波电压可以写成如下形式:
\begin{equation}%3-1
b_1=S_{11}a_1+S_{12}a_2
\end{equation}
\begin{equation}%3-2
b_2=S_{21}a_1+S_{22}a_2
\end{equation}
$S_{11}$,$S_{12}$,$S_{21}$,$S_{22}$为散射参量,这四项的定义分别为:
\begin{itemize}
	\item $S_{11}$ 为端口 2 匹配时,即($Z_L=Z_0$)时,端口 1 的反射系数
	\item $S_{22}$ 为端口 1 匹配时,即$(Z_S=Z_0)$时,端口 2 的反射系数
	\item $S_{12}$ 为端口 1 匹配时,端口 2 到端口 1 的传输系数
	\item $S_{21}$ 为端口 2 匹配时,端口 1 到端口 2 的传输系数
\end{itemize}

S-parameter 的物理意义是在相应的端口接匹配负载的情况下所对应的反射系数或者是传输系数。在微波系统中,一般情况下是采用对数的形式来表示 S-parameter 的四个参量:
\begin{equation}%3-3
|S_{11}|_{dB}=10\log|S_{11}|^2=20\log|S_{11}|
\end{equation}
\begin{equation}%3-4
|S_{22}|_{dB}=10\log|S_{22}|^2=20\log|S_{22}|
\end{equation}
\begin{equation}%3-5
|S_{21}|_{dB}=10\log|S_{21}|^2=20\log|S_{21}|
\end{equation}
\begin{equation}%3-6
|S_{12}|_{dB}=10\log|S_{12}|^2=20\log|S_{12}|
\end{equation}

二端口微波网络是微波网络分析和网络综合的基础,在重组和简化复杂电路以及深入研究有源以及无源器件的特性方面,已经成为非常重要的工具。

本课题设计的第一步就是要分析所选用晶体管的 S 参数,先利用 ADS 软件进行扫描,在选定晶体管的直流工作点以后,就可以进行 S 参数的扫描了,由于 Ka 波段频率过高,大部分高频段的低噪声放大器生产厂家只提供了 s2p 文件,因此本设计同样采用 s2p 文件进行扫描,首先建立一个工程并命名为S$\_$params,如图\ref{pics/3-16}所示:
\pic[htbp]{S$\_$params工程建立图}{width=0.7\linewidth}{pics/3-16}

然后在 Schematic Design Temples 栏中选择“S-Params”,如图\ref{pics/3-17}所示。
\pic[htbp]{S 参数扫描模板选择}{width=0.7\linewidth}{pics/3-17}

单击 OK 以后,即可生成新的原理图,如下图\ref{pics/3-18}所示,显示的是一个 ADS 自带的 S 参数仿真的模版:
\pic[htbp]{S 参数扫描模板}{width=0.7\textwidth}{pics/3-18}

然后下一步就是将 s2p 文件导入 ADS 中,同样是在原理图中进行操作,在左边的下拉框中选择 Data Items,如图\ref{pics/3-19}:
\pic[htbp]{s2p 文件的选择}{width=0.7\textwidth}{pics/3-19}

用记事本或者是其他的软件打开此文件,注意开头的注释部分会含有端口信息,也即是关于文件的信息,如本课题中的即是二端口网络,一般情况下 s2p 文件均为二端口参数,如下图\ref{pics/3-20}所示。
\pic[htbp]{s2p 文件的信息}{width=0.7\textwidth}{pics/3-20}

再次切回到 ADS 的原理图当中,在整个原理图的左侧 Palette 列表中选取相对应的元件。由于本次设计为 s2p 文件(一般为二端口),选择中间数字为 2 的图标,将其放在原理图中,然后双击 s2p 文件,选择文件的路径,其余的设置均保持不变,如图\ref{pics/3-21}所示:
\pic[htbp]{导入 s2p 文件}{width=0.7\textwidth}{pics/3-21}

将 s2p 文件和 S 参数扫描模版按照下图连接,同时注意将 s2p 文件接地。如图\ref{pics/3-22}所示:
\pic[htbp]{s2p 文件扫描电路}{width=0.7\textwidth}{pics/3-22}

最后需要对S参数仿真模版中的频率扫描范围进行更改。双击模版中的 S 参数仿真控制器,会弹出参数设置窗口,在参数设置窗口中对参数进行如下的设置,如图\ref{pics/3-23}所示:
\begin{itemize}
	\item Start=30GHz,表示扫描的起始频率从 30GHz 开始。
	\item Stop=42GHz,表示扫描的频率到 42GHz 终止。
	\item Step=0.1GHz,表示在扫描过程中,频率间隔为 0.1GHz。
\end{itemize}
\pic[htbp]{扫频范围的设置}{width=0.7\textwidth}{pics/3-23}

设置好的S参数仿真空间如图\ref{pics/3-24}所示:
\pic[htbp]{参数扫描电路}{width=0.7\textwidth}{pics/3-24}

连接好原理图,以及设置好各个参数以后,即可以进行参数扫描仿真了,点击 Stimulate-Start Stimulate,仿真以后得到如图\ref{pics/3-25}结果:
\pic[htbp]{参数扫描结果}{width=0.7\textwidth}{pics/3-25}


数据显示窗口中除了有扫描曲线表示出S(1,2)和S(2,1)的结果以外,也同时利用史密斯圆图表示了数据S(1,1)和S(2,2)的结果。在微波频段当中,史密斯(Smith)圆图为分析传输线的问题和关于匹配电路的问题提供了一个非常有实际应用的图形化设计和分析工具,并且十分方便有效。通过归一化阻抗和导纳 Smith 圆图,可以很容易且快速设计出实现微波放大器最佳性能的匹配电路\citeup{11}。Smith 圆图基本上是一个所有无源阻抗都画在以单位长度为半径之内的反射系数圆图。从 Smith 圆图中所能够读出数据的精确度,完全足够解决实际应用中的设计问题。通过归一化阻抗和导纳 Smith 圆图,可以很容易并且十分快速地设计出实现微波放大器最佳性能的匹配电路。Smith圆图是在反射系数平面中表述所有Re[z]$\geqslant$0的Z值与反射系数的关系,该平面叫做$\Gamma$平面,其表达式为:
\begin{equation}%3-7
G=\frac{Z-Z_0}{Z+Z_0}
\end{equation}
$Z_0$ 是传输线的特征阻抗或者参考阻抗的特征值。定义归一化阻抗为:
\begin{equation}%3-8
Z=\frac{Z}{Z_0}=\frac{R+jX}{Z_0}=r+jx
\end{equation}
因此有:
\begin{equation}%3-9
\Gamma=\frac{Z-1}{Z+1}
\end{equation}
下图\ref{pics/3-26}中表示了Z取不同值的变换特性。例如,Z=50 欧姆且 Z$_0$=50 欧姆,则 z=1,而$\Gamma$=0。也就是说,归一化的 z 平面 z=1 的点映射到 $\Gamma$ 平面的原点。同样,z=0 映射到$\Gamma$=-1。
\pic[htbp]{史密斯圆图的由来}{width=0.7\textwidth}{pics/3-26}



另外,Smith 圆图具有阻抗圆图和导纳圆图两种形式,可以视具体不同情况下来选用。阻抗和导纳之间的变换还可以通过重叠两个 Smith 圆图并且通过将其中一个图旋转 180 度来实现。旋转的圆图表示导纳,另一个表示阻抗。原来的圆图和旋转的圆图重叠起来统称归一化阻抗导纳坐标 Smith 圆图。如图\ref{pics/3-27}所示。
\pic[htbp]{史密斯阻抗圆图}{width=0.7\textwidth}{pics/3-27}

由上图中我们可以看到,在 Smith 阻抗圆图中存在等电阻圆、等电抗线、纯电阻线、电感平面(j$\omega$L)、电容平面(1/j$\omega$C)、开路点、短路点和 50 欧姆点等等。当然,在相对应的在导纳圆图中也存在等电导圆和等导纳线等。

\section{低噪声放大器稳定性的分析}
微波低噪声放大器的设计之所以要进行稳定性的分析是因为放大器如果工作于不稳定的状态的时候,放大器电路可能会发生振荡,有源器件可能将会进入大信号工作状态,放大器的性能随之也会发生比较显著的变化,当这个时候通常放大器的小信号 S 参数将变得无效,因此电路设计也会因此变得不正确。另外当放大器出振荡情况的时候,会使电路产生更大的噪声从而降低电路的性能,振荡过大的时候放大器晶体管可能会毁坏导致放大器不能继续正常工作。至于放大器的自激振荡可以分为两种,分别为低频振荡和高频振荡,高频振荡通常是由于部件接地不良,或者是介质基板太厚而导致寄生电感量过大、晶体管的引线电感太大形成的负反馈造成的;而低频振荡通常是由于供电的偏置电路电源滤波网络设计不好而造成的。

放大器的稳定性取决于晶体管的 S 参数,匹配网络以及端置条件。在二端口网络当中,当输入或输出端口呈现负阻时,有可能会使放大器的电路产生振荡。对于放大器的输入和输出端口不存在负阻的情况,放大器则是无条件稳定。由输入阻抗和输出阻抗的实部均大于 0,可以知道放大器无条件稳定必须满足输入、输出反射系数均小于 1。一般可以用二端口网络来表征放大器,网络的外部终端条件比如$\Gamma_L$和$\Gamma_S$,以及 S 参数可以确定这个网络。网络如果要稳定,意味着反射系数的模小于1\citeup{12}。对于各种无源负载和信号源的阻抗,如果网络的输入、输出阻抗实部均大于零,在给定的频率范围内绝对稳定,否则网络将有不稳定的可能性。一个二端口网络是否稳定可以使用稳定系数K来表示。一个绝对稳定的二端口网络必须满足$K> 1$,并且$|\Delta|<1$,其中:
\begin{equation}%3-10
K=\left(
1-|S_{11}|^2-|S_{22}|^2+|\Delta|^2
\right)/2|S_{11}S_{22}|^2
\end{equation}
\begin{equation}%3-11
|\Delta|=|S_{11}S_{22}-S_{12}S_{21}|
\end{equation}
如果分析出放大器不稳定的话,可以试图通过以下的方法来改善晶体管的稳定性。对于,非绝对稳定的放大电路,改善晶体管稳定性的方法主要可以采用以下几种:(1)源级负反馈法,从噪声角度考虑一般情况下会使用无耗感抗负反馈,在实际电路中,反馈元件通常采用微带线来构成。试验表明适当的源级串联感抗负反馈,使得输入、输出驻波略有减小,稳定性增强。但是反馈强度不能过强,否则稳定性反而下降。(2)采用铁氧体隔离器,在潜在不稳定时,加入隔离器使其源和负载都达到标准阻抗,使得器件绝对稳定。(3)稳定衰减器,可以在漏极串联电阻或fl型阻性衰减器,通常接在低噪声放大器末级或末前级输出口。(4)当放大器频带外增益出现不易消除的增益尖峰时,比如在工作频带外的低端,可以使用低端增益衰减网络。具体操作应该根据晶体管和电路特性选择。

\section{输入输出微带线匹配的设计}
S参数扫描分析完成以后,可以进行输入输出匹配的设计。目前所应用的微波集成电路有两种,第一种叫做集总参数电路,在集总参数电路中,主要应用电感、电容、电阻等电路元件,其尺寸远远小于工作波长,借助淀积、蒸发、光刻等工艺印制在介质基片上面,然后再和有源微波固体器件连接以后即构成整个微波集成电路。第二种就是本设计即将用到的分布参数型集成电路或微带集成电路(简称为微带电路),电路中的元件由分布参数的微带线构成,主要包含了按照设计图形印刷在介质基片一面的导体带条和另一面的金属接地,图形的尺寸几乎可以和工作波长相比拟,和微波固体器件相连接后即构成了整个微带电路。选择第二种匹配的原因主要是有以下几点:第一个是因为比较两种微波集成电路,第一种的工作频带比较宽,集成度比较高,但是其工艺是比较复杂的,质量不容易得到保证,而且电路元件的精度很难提高,所以会导致整个电路的一致性比较差;相反对于微带电路只要可以保证精确的印制工艺(这点相对是很容易做到的),就能够获得较高的电路质量,因此在目前的实际当中,绝大多数会选择这种电路。第二个原因是微带线可以印制在很薄的介质基片上,可以薄至 1mm 以下,因此它的横截面尺寸比波导、同轴线要小得多。它的纵向尺寸可以和工作波长相比拟,但是因为可以采用高介电常数的介质基片,使得线上的波长比自由空间的波长小了几倍,同样可以减小。另外,整个微带电路元件公用接地版,只需要由导体带条构成电路的图形,使整个电路的结构变得更加的紧凑,也正是因此,微带电路可以更好的解决小型化的问题,与波导、同轴线元件相比,大大的减小了重量和体积。

特性阻抗 $Z_0$ 和相速 $V_\phi$ 是任何微波传输线的两个主要参量。第一个与阻抗匹配有关,第二个决定了传输线点长度和几何长度的关系。微带线是准 TEM 模式,它的相速,传播常数和特征阻抗可以由静态和准静态解来获得。微带线是微波传输线的一种,是一种目前比较流行的平面传输线,最主要的原因是因为它可以用照相印刷工艺来加工,并且很容易与其他无源和有源的微波器件集成。特性阻抗是传输线上的行波电压和行波电流或者是入射波电压对入射波电流之比;相速所表示的是电磁波在传输线上的行进速度。由于波的速度系以等相位点向前移动的速度表示,因此又称为相速。当传输线的分布电感和分布电容已知后,可以根据公式求得 $Z_0$ 和 $V_\phi$。在计算这两个参量的时候,近似的认为微带线工作于 TEM 波,即把微带线横截面的电场和磁场的分布看成于类似静场,完全按照静场的一整套方法计算出空气微带线和介质微带线的分布电容,并且由此计算出有效介电常数和特性阻抗。由于介质的充填程度与微带线横截面几何参量 W/h 有关,也就是随着 W/h 的变化而变化,因此微带线的相速度和特性阻抗都是横截面几何参量 W/h 的函数。对于不同介质基片的相对介电常数、不同横截面积几何参量W/h的和特性阻抗 $Z_0$ 的具体数据。低噪声放大器设计中,输入、输出匹配网络的设计是在获得噪声系数、增益指标的前提下将晶体管的输入阻抗、输出阻抗分别变换到标准的SOS2。由于晶体管的最佳噪声匹配阻抗和最小输入驻波的阻抗一般都不相同,低噪声放大器的最大增益和最佳噪声系数不可能同时获得,在实际设计过程中一般都采取指标折中的办法\citeup{13}。本次设计输入输出均采用阻抗为 50 欧姆的微带线匹配。

进行匹配首先要做的第一步是计算微带线的几何参量,由于本次设计所使用的软件为 ADS,我们可以利用软件本身自带的计算工具,即利用 LineCalc 计算微带线参数。

首先是打开 ADS 软件,可以利用刚刚创建的 Workspace,在刚刚扫描 S 参数的原理图中进行操作,在原理图的设计窗口中,执行菜单命令 Tools-LineCalc-Start LineCalc,即可以弹出 LineCalc 窗口,界面如下图\ref{pics/3-28}所示:  
\pic[htbp]{微带线计算参数设置界面}{width=0.7\textwidth}{pics/3-28}



在本次设计中,主要需要改动的参数在公共参数显示窗口(Shared Parameters Display),和参数显示窗口(Parameters Display),公共参数显示窗口中包含了两个主要的部分,第一部分是介质基片参数设置栏,第二部分是元件参数设置栏。在第一部分中有几下一个内容:
\begin{itemize}
	\item ID= MSUB$\_$DEFAULT,表示微带线默认的介质基片参数。
	\item Er为微带线介质基片的相对介电常数,本设计中Er=2.2。
	\item Mur=1,表示微带线介质基片的相对磁导率为1。
	\item H为微带线介质基片的厚度,本设计中为0.254mm。
	\item Hu表示微带线的封装高度,一般指屏蔽盖。
	\item T为微带线金属片的厚度。
	\item Cond表示金属片的电导率
	\item TanD为微带线损耗角正切。根据Rogers5880的参数,在本设计中 TanD=0.0009。
	\item Rough=0mil,表示了微带线的表面粗糙度为0mil。

\end{itemize}
在第二部分元件参数设置栏中:
\begin{itemize}
	\item Freq表示微带线的工作频率,在本设计中Freq=36GHz。
	\item Wall 1=默认值,表示条带H的边缘到第一侧壁的距离,默认值为1.0e+30mil。
	\item Wall 2=默认值,表示条带H的边缘到第二侧壁的距离,默认值为1.0e+30mil。
\end{itemize}
在参数显示窗口中,同样包含了两部分。第一部分是物理尺寸参数设置栏:
\begin{itemize}
	\item W:表示微带线金属片的宽度。
	\item L:表示微带线的长度。
\end{itemize}
第二部分为电尺寸参数设置栏:
\begin{itemize}
	\item Z0:表示微带线的特性阻抗,在本设计中取值为50Ohm。
	\item E$\_$Eff:表示微带线的点长度
\end{itemize}

这两部分的关系为,如果电尺寸数据确定,计算物理尺寸参数则单击\linebreak[4] 【Synthesize】 则可以得到结果;如果物理尺寸数据确定,计算电尺寸参数则可以单击\linebreak[1] 【Analyze】 按钮以得到结果。LineCalc 窗口的底部的状态栏显示变量数据的状态。当没有 “Synthesize” 或者 “Analyze” 的时候,状态栏显示的是 “Values are not consistent”;当 “Synthesize” 或者“Analyze”后,状态栏显示的是“Values are consistent”。改动参数以后如图\ref{pics/3-29}所示:
\pic[htbp]{微带线计算参数设置}{width=0.7\textwidth}{pics/3-29}

然后进行计算,点击【Synthesize】可以得到微带线的几何参数。即在下一步画版图的时候微带线的具体尺寸。

\section{直流偏置电路的设计}
偏置电路一般情况下为低通滤波器网络,主要起到两个作用,第一个作用是可以隔离射频信号进入电源电路,以防止射频信号对电源产生干扰。第二个作用是,与此同时又能够滤除微波晶体管产生的各种谐波级高频信号。偏置电路采用/4高阻线加扇形线的结构,要实现在所要求的频段内能够很好的遏制交流信号,将交流信号对射频电路的影响尽可能的降到最低,同时又保证不影响射频信号的正常传输。另外直流偏置电路的设计目标是为了晶体管选择适当的静态工作点,因为静态工作点基本决定了晶体管的整个工作的状态,因此它的设计是非常重要的。微波单片电路中的偏置电路能起到提供偏置电压、去耦滤波和隔离信号等作用。虽然前文分析己经确定了晶体管的静态工作点,但考虑到在实际的电路当中外围加电和后期对于版图的仿真,同样也还需要进行偏置电路结构的设计和仿真。放大电路中的偏置电路由外加电源方式的不同可以分为单电源供电和双电源供电。单电源供电就是只提供一个外加电源,通过电阻分压来使栅极和漏极达到一个合适的工作电压。而双电源供电则是提供两个外加电源,分别加到晶体管的栅极和漏极。其中单电源供电一般主要是用于较低的频段并且对噪声系数要求不是很高的放大电路当中,因为虽然它的结构比较简单,但是由于电阻的引入对噪声会有一定的影响。所以相比之下,双电源供电在高频段的低噪声放大器的设计中有比较大的优势。所以在本次设计当中选用第二种,双电源为低噪声放大器供电的方式。为此要分别设计栅极跟漏极偏置电路。漏极偏置电路的作用除了为低噪声晶体管提供恒定的漏极正电压和漏极电流之外,还要隔离栅极到漏极的信号,栅极偏置电路的功能是提供恒定的栅极负电压之外,另外还有稳定电路的作用,遏制漏极到栅极的信号。综合以上各个方面的要求,在本课题中决定采用TGA4508资料手册中推荐的偏置电路来进行设计。如下图\ref{pics/3-30}所示:
\pic[htbp]{资料手册推荐偏置电路}{width=0.7\textwidth}{pics/3-30}

为了使整个电路有较好的抗干扰性,在本次设计中,需要在输入输出各加两个旁路电路。

因为旁路电容作用是把射频能量从漏极电压引入地,所以旁路电容的选择很重要。一般情况下是选择自谐振频率在信号的频率附近的电容。在默认值的附近调节电容值会发现电容值的大小对低噪声放大器的性能基本没有影响,则说明此时容值己经足够大,因此直接选用默认值就能起到旁路的作用。本次设计选用200pF的电容。 

\section{同轴-微带过渡结构的设计}

在射频测试系统中,同轴到微带的过渡结构很常见,例如,为了测试印制电路板(Printed-Circuit Board)上电路的电性能,方便电路与测试设备相连接,通常需要在电路的输出端口处焊接一个标准的50欧姆的同轴接头,即将同轴接头内导体与微带焊接到一起,同轴接头的外导体与微带的地板平面焊接于一起,这样便形成了一个典型的同轴到微带的过渡结构。对于这些过渡电路,一般都需要反射系数小、过渡的插损低、带宽宽,同时电路结构要简单,便于装配\citeup{14}。 

从同轴线所传播的电磁波模式来看,同轴线传播的是横电磁波,而微带线传播的是准横电磁波,两种不同的传输线形式在连接时,即在同轴接头与微带的连接处,将会产生不连续,并且随着频率的不断增高,这一影响将会更加的明显,例如导致较大的回波损耗或者是插入损耗等。因此在设计较好的同轴到微带的过渡结构时,必须应用一些补偿措施,来消除或者是减小这种不连续所造成的影响,使得这个过渡的结构具有较小的回波损耗及插入损耗,以确保过渡结构的引入对待测电路不会造成太大的影响。

在毫米波频段,为了便于测试,天馈以及独立微带电路之间的连接,常常需要将微带电路输入、输出端口通过转换结构过渡到矩形波导。在需要将信号作一段距离传输时,也必须将电路从微带转换成波导,以降低传输的损耗。因而采用微带的毫米波集成电路往往都必须具有宽带特性的波导-微带过渡的接口。探针结构是工作于TEM波的同轴线和工作于TE$_10$模的矩形波导间的一种比较常用的过渡结构。在实际的应用中,过渡器的一个不可以被忽视的一个附加条件是对于气密性的要求。很多的微带电路,特别是军用的微带电路,为了能够保证在各种恶劣的环境下都能够维持正常的工作状态,性能保持稳定,因此对系统的气密性就提出了更高的要求,而通常所采用的E面探针型波导-微带过渡结构在波导上的开口较大,不仅仅使得系统的气密性收到了很严重的影响,对矩形波导内的场分布也起到了较大的扰动。综合的利用相关的理论知识并且考虑到了波导-微带结构过渡结构的具体要求,采用波导-同轴探针-微带过渡结构能够获得较好的效果。

探针在波导中相当于一个小天线,若同轴线接波源,探针便是发射小天线,它向波导所限定的辐射电磁波。一般情况下来说,只要电磁波的电场与波导某模式的电场或者磁场的分量一致,该模便会被激励。如下图所示,采用 R.F.Harringon 的等效电路法进行探究。在矩形波导内,由同轴波导驱动一根很细的探针,细探针被放置在z=0的波导模截面上,它的一端由金属组成短路面,另一端接匹配负载。矩形波导的宽边为a,窄边为b,Ls是短路面到同轴探针的距离。探针沿y轴的方向放置,伸入波导的部分高度为h,与y轴的距离为d(即在X轴上的坐标为x$_1$)。从平面z=0来看,Z$^+$和Z$^-$分别是在+z和-z方向矩形波导的输入阻抗。X是存在于探针附近的高次模电抗。

根据公式可以推导出一系列的结果,将探针通过波导宽边开口插入波导腔中,通过一段起耦合探针作用的探针把波导中的电场耦合到同轴线当中去。在HFSS中建立模型,如下图\ref{pics/3-31}所示:
\pic[htbp]{波导到同轴线的过渡结构}{width=0.7\textwidth}{pics/3-31}


\section{版图的绘制}
在完成电路原理图的仿真以后,则可以根据初步绘制的低噪声放大器原理图进行版图的设计。版图设计在整个微波单片集成电路设计中是十分关键的一个步骤,它直接反映了最终微波单片电路产品性能的好坏。由于在本课题中我们设计的低噪声放大器工作频率较高,此时的寄生效应和耦合效应是不可以被忽略的,所以为了使最终仿真结果更加准确和可靠,在版图仿真时要整体考虑芯片的面积,并且需要合理地规划电路的布局布线。

版图在整个射频电路的设计中是十分重要的,在实际的设计过程当中,有两个方面要格外的注意,第一点是走线,第二点元件的放置。

关于第一点,走线指的是把两个元件连接到一起,在射频电路当中,信号走线就是微带线,因此可知,长的走线就是长的微带线,它对电路的性能可能对电容、电感以及电阻更加的重要。在一些射频电路的设计当中,电路主要就是有器件和微带线组成。因此严格地处理走线的问题,是保证整个整个射频电路设计成功的重要基础。

在版图的设计当中,四分之一波长是非常重要的参数,。经过四分之一波长的走线,射频信号可能会发生状态变化,从短路状态变成开路状态,或者是从零阻抗变到无限大的阻抗。在版图的设计规划当中,走线的长度要尽可能的短,即走线的长度远远的小于四分之一波长。如果走线的长度相当于或者是大于四分之一波长,在进行电路仿真的时候必须要作为一个元件来进行对待,在实际设计当中,走线的宽度通常为5μm$<$W$<$100μm。

至于走线的风格,第一是走线的长度,不能够单纯的去追求美观,向数字电路设计那样,一整束走线都相互平行,这样的走线风格对整个电路的性能来说是十分糟糕的。既增加了走线的长度,而且会在走线中引入极大的串扰,另外平行的走线也会引入互容和互感,在下图\ref{pics/3-32}中,第一种即偏向美观,第二种是比较好的走线方式:
\pic[hp]{两种不同的走线方式}{width=0.7\textwidth}{pics/3-32}


第二是走线的拐角问题,下图\ref{pics/3-33}中有三种走线拐角的方式,在实际设计当中,第一种是优于第二种和第三种的:
\pic[hp]{三种不同的走线拐角方式}{width=0.7\textwidth}{pics/3-33}


主要的原因以下几个,第一是因为它是平滑的,因为在走线当中,任何角度,特别是急拐弯的角度,都会在电磁场中产生奇异点并且引入相当大的辐射。第二个原因是因为它是三种走线方式当中最短的一种。

第三个需要注意的问题是:要优先画相邻的走线,首选画的是相互垂直的两条相邻走线,并且尝试画尽可能不是平行的走线。 

综合以上,走线的时候格外需要注意以下三点:
\begin{enumerate}
	\item 走线尽可能地短。
	\item 尽可能地平滑。
	\item 走线尽可能地正交。
\end{enumerate}

第二方面要注意的是元件。在元件中最先要注意的就是器件,对于多个器件的摆放,要尽可能的采用“方阵”的形式摆放,而不是摆成一条直线的形式。然后在摆放电感的时候要注意两个电感之间要有一定的间隔,而且要添加保护环,保护每一个电感。在摆放电阻的时候,由于电阻在所有器件中相对容差是最大的,所以为了降低容差,我们选择采用并联的方式来进行排列,如果想要的到一个一千欧姆的电阻,可以选择采用是个十千欧姆的电阻并联的方式得到。

至于芯片的多余空间,有一些设计版图喜欢用接地金属覆盖芯片所有的剩余空间,认为接地的部分有利于元件间或者是模块间的隔离,或者是有利于保护模块以免受到其他模块的干扰。但是实际上,这些金属层是有边界效应的;模块间的串扰被加强,元件间的干扰也会随之增加,这是因为:
\begin{enumerate}
	\item 器件和接地面之间杂散电容增加。
	\item 回流分布在更大的接地面上,以至于模块间的串扰被加强。
\end{enumerate}

所以,在射频电路版图中,留出一些自由的空间是比较必要的,不用尽可能的用接地金属覆盖所有的剩余空间,也不要试图把所有的元件都挤压到一起。

在本次设计的实际操作中,我们设计中首先根据之前画好的低噪声放大器的原理图自动生成版图,(添加生成过程和图)但是这个时候所得到的版图还是比较混乱的,存在很多问题,例如微带线过长、器件摆放不合理、匹配网络尺寸不合适等问题。特别是有的微带线间距太小,而有的微带线太长导致芯片的尺寸过大,一般情况下会将微带线进行折叠处理,但折叠之后的微带线的特性会发生变化,因此需要调节折叠后微带线的长和宽,使它的特性和折叠之前达到一致。因此需要运用 Momentum 电磁场仿真同步修改原理图,使原理图和版图仿真结果达到同步并且更新电路参数,这是一个反复的过程,需要比较长时间的验证和加以修改。在保证和原理图各个无源部分性能达到一致的同时,还要考虑尽可能地缩小芯片的尺寸,这也需要反复地调整器件的位置以及微带线的折叠情况。在还没有做调整的情况下,初步生成的电路版图如图\ref{pics/3-34}所示:
\pic[hp]{放大器布线}{width=0.7\textwidth}{pics/3-34}


在画出版图以后,有关放大器的实现仍需要注意几个问题,不仅在设计中要定量考虑具体因素使毫米波低噪声放大器实际电路能达到设计的技术指标,而且还要在加工、组装和测试过程中避免非理想因素的引入,主要注意一下几个方面:
\begin{enumerate}
	\item 接地方面。对于MMICs器件其背面既是直流地又是射频地,分立器件的源级必须接地良好,在设计的时候要让这些接地点与腔体直接相连以尽量减小接地阻抗,避免因为接地不良而引入的反馈导致放大器指标的恶化。
	\item 输入输出连接方面。因为低噪声放大器的输入端的反射与损耗都对噪声系数有很大的影响所以放大器的输入端必须连接良好。用金丝连接输入输出是毫米波频段常用的方式但是不能忽视金丝电感在毫米波频段内对电路性能的影响。为了减小金丝电感的影响主要方式是尽量缩短金丝的长度或者采用多跟金丝的连接方式。此外还需要在电路调试与测试过程注意保证输入输出阻抗与设计的一致性否则会恶化电路噪声参数或者引起自激。
	\item 屏蔽方面。由于低噪声放大器的信号电平比较低所以容易受到外部信号或者其他大信号电路的干扰而这些干扰引入电路之后也会恶化电路的参数,因此在电路腔体时要有良好的屏蔽措施以减小上述干扰信号对于放大器的影响。
	\item 隔离方面。通过屏蔽腔内空间电磁辐射耦合或者表面波耦合,不同电路之间或单级放大器输入输出间会出现反馈。当增益较高时,较小的反馈也可能产生噪声包,处于带内时使噪声系数恶化,严重时会使放大器自激。通过级间隔离可减少寄生耦合。例如:在级间和单级输入输出之间加金属隔板,减小屏蔽腔尺寸,对工作频段形成截止波导可衰减空间辐射。将多级放大器的基片分离,在传输线测制作周期性光子带隙结构,可抑制表面波的传输。这些措施可提高输入输出和级间隔离度,保证放大器稳定工作。\\
	在装配的过程中,绝缘子需要烧结在相应的孔中,还需要将绝缘子的一端链接到放大器电路的馈电点处,射频电路基片使用导电胶粘在腔体上即可。另外,由于使用的是管芯,还需要在管芯的漏极跟栅极与匹配电路的微带线使用金丝键合。考虑到一根金丝键合可能会带来额外的祸合寄生参数并且对电路的性能影响较大,此处采用两根金丝键合的方式。管芯本身有接地孔,所以事先需要切掉安装管芯位置的基片,让管芯直接与腔体接触即可实现源级接地\citeup{15}。
\end{enumerate}



