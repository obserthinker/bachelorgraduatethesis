% !Mode:: "TeX:UTF-8"

\chapter{低噪声放大器的理论基础}
\section{低噪声放大器的概述}

低噪声放大器的主要作用是可以放大从空中接收到的微弱信号,并且减少噪声干扰,提供系统解调出所需要的信息数据。现如今人们对各种无线通信的工具要求越来越高,例如有几个很典型的如:要求辐射功率要小,作用的距离要远,覆盖的范围要广等等。这样就对接收灵敏度提出了更高的要求,系统接受灵敏度的计算公式如下:

\begin{equation}%2-1
S_{min} = -144(dBm/Hz)+NF+10lgBW(MHz)+S/N(dB)
\end{equation}

从式子中可以看出,当系统的带宽和信噪比确定时,此时对于系统的灵敏度起决定性作用的就只有噪声系数,所以低噪声放大器的设计对整个接受系统是很重要的,并且提高灵敏度是关键的手段之一。从上述分析中可以知道低噪声放大器(Low Noise Amplifer)是收发组件中很重要的,几乎可以称为是核心部分,一般情况下是直接与天线相连,由于它的位置是处在接收机的最前端,所以对它有较高的要求,最基本的要求是提供一定的增益,并对信号进行放大,与此同时,还要有较低的噪声系数和很好的线性度。因为低噪声放大器的噪声系数以及性能如果稍微有一点恶化,对整个电路都有着非常大的影响,所以设计出性能比较好的低噪声放大器至关重要。

作为接受系统前端的重要部分,低噪声放大器主要有以下特点\citeup{9}:
\begin{itemize}
	\item 为了尽可能的减小后级系统的信噪比,要求噪声系数的同时还要求有足够的增益以保证后级系统能够正常工作,但是为了防止后级电路产生非线性失真,增益又不能太大,而且还要保证放大器处于稳定状态。
	\item 应该有足够大的线性范围以抑制其他强于从天线接收的微弱信号对于系统的影响。
	\item 放大器输入端必须实现很好的匹配以减小输入驻波的同时实现最小噪声系数。
	\item 为了抑制诸如镜像频率或者带外干扰,一般具有选频特性。
\end{itemize}

如图\ref{pics/2-1}为一般情况下低噪声放大器结构图:
\pic[htbp]{低噪声放大器结构图}{width=1\textwidth}{pics/2-1}

\section{低噪声放大器的基本理论}

由于放大器在实际应用中,等效电路过于复杂,实用性不是特别高,所以一般情况下在毫米波段采用S参数来描述有源器件,将有源器件看成一个二端口网络,网络特性通过S参数和噪声参量来描述,一般情况下厂家会给出s2p文件,内含参数,如果厂家没有给出,也可以通过自行测量获得。常见典型的HEMT二端口网络如图\ref{pics/2-2}所示:
\pic[htbp]{典型的HEMT两端口网络}{width=1\linewidth}{pics/2-2}

在上图中,网络S参数以及归一化入射与反射波的关系如下: 

\begin{equation}%2-2
b_1=S_{11}a_1+S_{12}a_2
\end{equation}

\begin{equation}%2-3
b_2=S_{21}a_1+S_{22}a_2
\end{equation}

在上式中,$a_1$与$a_2$为二端口网络的归一化入射波,网络的归一化反射波由,来表示。

$\Gamma_{in}$表示负载$Z_L$的输入反射系数,同时$\Gamma_{out}$表示任意源阻抗下的输出反射系数:

\begin{equation}%2-4
\Gamma_{in}=S_{11}+S_{12}S_{22}\Gamma_L / (1-S_{22}\Gamma_L)
\end{equation}

\begin{equation}%2-5
\Gamma_{out}=S_{22}+S_{12}S_{22}\Gamma_S / (1-S_{11}\Gamma_S)
\end{equation}

在式子中:$S$表示源阻抗的反射系数,$L$表示负载端的反射系数。于是有下式:

\begin{equation}%2-6
b_1=\Gamma_{in} a_1
\end{equation}

\begin{equation}%2-7
a_1=a_s+\Gamma_S b_1
\end{equation}

\begin{equation}%2-8
a_2=\Gamma_L b_2
\end{equation}
在式子中,$a_S$表示信号源输出的归一化入射波,上述的二端口的网络的实际输入功率为$P_{in}$二端口网络在在信号输入端实现共轭匹配的时候获得的信号源输入功率最大,用资用功率来表示:
\begin{equation}%2-9
P_{in}=\left( \frac{1}{2}|a_S|^2 \right) / \left(1-|\Gamma_S|^2\right)
\end{equation}
用晶体管等效二端口网络,整个网络的噪声系数对于任意源阻抗可以表示成:
\begin{equation}%2-10
N_f = N_{fmin}+\frac{R_n}{G_S}\left[ \left(G_S-G_{opt}\right)^2 +
\left(B_S-B_{opt}\right)^2
 \right]
\end{equation}
其中,信号源的输入导纳用$Y_S=G_S+jB_S$来表示,为最佳噪声系数,表示最佳源导纳,为等效噪声电阻,表示噪声系数对信源导纳的敏感度。源反射系数和最佳源反射系数也可以表示成:
\begin{eqnarray}%2-11
\begin{aligned}
\Gamma_S &= (Y_0-Y_S)/(Y_0+Y_S)\\
\Gamma_{opt}& = (Y_0-Y_{opt})/(Y_0+Y_{opt})
\end{aligned}
\end{eqnarray}
那么噪声系数$N_f$就可以写成:
\begin{equation}%2-12
N_f=N_{fmin}+4\frac{R_n}{R_o}\left|\Gamma_S - \Gamma_{opt}\right|^2/
\left[
\left|1+\Gamma_{opt}\right|^2
\left(1-\left|\Gamma_S\right|^2\right)
\right]
\end{equation}
在一般情况下为50$\Omega$,$N_{fmin}$,$R_n$和$\Gamma_{opt}$和是晶体管的等效噪声参数,随着频率的变化而变化。

在微波网络中,二端口微波网络是最基本的。表征二端口网络特性的参量可以分为两大类,第一类为反映参考面上电压与电流之间关系的参量,包括了阻抗参量、导纳参量、转移参量;第二类网络参量是反映参考面上入射波电压与反射波电压之间的关系的参量,包括了散射参量和传输参量。各种参量对应着不同的矩阵形式。其中导纳矩阵为阻抗矩阵的逆矩阵,转移参量对应着ABCD矩阵,传输参量对应着T矩阵。转移参量和传输参量在多个二端口网络级联方面有着重要的应用\citeup{10}。

\section{低噪声放大器的技术指标}

一个低噪声放大器的性能主要是包含噪声系数、增益和稳定性等等。这些指标都是对整个系统的设计都至关重要的,每一项都会影响放大器的性能。

\paragraph{1. 工作频率和带宽} 工作频率指的是放大器能正常工作的范围,从fL到fH,即从下限频率到上限频率,其中正常工作指的是在整个频段噪声系数满足要求,同时功率增益也满足增益平坦度的要求。

常规带宽的定义为$B=f_Lf_H$,相对带宽偶尔也会用到,定义为$B=(f_Lf_H)/f_o$,在公式中$f_o$为中心频率,即放大器工作频率的中点。一般情况下,低噪声放大器的工作频带不太宽,相对带宽最多一般为20\%,因为如果频带过宽,比较难获得较低的噪声系数,而噪声系数却是放大器设计首先要考虑的指标。

\paragraph{2. 噪声系数}
当一个放大器没有输入信号的时候,但是其输出端仍能输出较为微弱的信号,这是输出信号的功率就用噪声功率来定义,而噪声功率用噪声系数来衡量,噪声系数$NF$可以定义为:
\begin{equation}%2-13
NF=(S_{in}/N_{in})/(S_{out}/N_{out})
\end{equation}
在上面的公式中,$S_{in}$和$S_{out}$,分别为系统输入和输出端口的信号功率,$N_{in}$和$N_{out}$分别为系统输入和输出端口的噪声功率。

因为当信号通过放大器时,放大器产生的噪声会使整体的信噪比变坏,所以从物理上定义噪声系数可以解释为信噪比下降的倍数。噪声系数的单位为分贝,单位转换公式如下:
\begin{equation}%2-14
NF=(dB) = 10\lg(NF)
\end{equation}

在实际设计的过程中,对增益的要往往都是超过一个晶体管的最大增益能力,因此需要多个晶体管级联,噪声系数的表达式为:
\begin{equation}%2-15
NF=NF_1+(NF_2-1)/G_1+(NF_3-1)/G_1G_2+\cdots
\end{equation}
即 Friis 公式,从这个公式中可以看出整体电路的噪声系数是由各级放大电路的噪声系数以及增益共同决定的。图\ref{pics/2-3}为 TGA4508 的噪声系数随频率变化的示意图:
\pic[htbp]{噪声系数变化图}{width=0.7\linewidth}{pics/2-3}
