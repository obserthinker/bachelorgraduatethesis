% !Mode:: "TeX:UTF-8"

\chapter{引言}

\section{课题的研究背景}

电磁波被人们认识发现以后,电磁场和微波技术到现在经过了一百多年的进步和发展,到如今微波己经广泛地被应用于雷达和卫星通信、导航和遥感、干扰和抗干扰以及医学和天文等众多应用领域\citeup{1}。早期的应用主要是集中在厘米和分米的波段,但是随着通信技术和国防事业的不断发展,近几年来,微波频谱拥挤的现象越发地显著,所以向频率更高的毫米波频段开发变得更为迫切。频率较高的毫米波主要有以下几个特点:(1)与分米波和厘米波相比较毫米波有着极宽的带宽。我们通常所说的毫米波一般情况下是指频率范围为26.5$\sim$300GHz的微波,带宽高达273.5GHz,超过毫米波之前所有带宽的10倍。由于大气层具有的损耗特性,远距离传播只能使用四个主要的“大气窗口”,但即使如此可用总带宽也可达135GHz,是毫米波以下各波段带宽总和的5倍。在如今频率资源十分紧缺拥挤的条件下无疑具有相当大的吸引力。(2)波束窄。对相同的天线来说辐射毫米波的波束比辐射米波或者分米波的波束要窄得多,因此更小的目标能够被识别,天线的分辨相应的变得更高。(3)毫米波元器件的尺寸通常都比较小,因此基于毫米波的系统想要小型化、平面化或者集成化也更加的方便。(4)与更高频率的红外及光波相比较,毫米波则具有全天候特性,传播受天气的影响要小很多\citeup{2}。正是由于毫米波具有以上的这些优势,再一个是在电子对抗中,宽频决定器件的抗干扰的能力,因此近几十年来毫米波技术和应用得到了十分迅猛的发展。低噪声放大器(LNA)一般处于接收系统前端,作用是在放大从天线接收下来的信号的同时又要保证对后级电路的信噪比影响尽可能地小,以确保整个系统的正常工作。低噪声放大器(LNA)的噪声性能对于整个接受系统的噪声以及灵敏度有很大的影响。所以如何设计性能优良的低噪声放大器(LNA)是微波毫米波收发系统设计的关键。

低噪声放大器(LNA)的设计,主要是要实现足够低的噪声系数以及足够的增益,此外还要动态范围较大,带宽较宽以及好的稳定性。还有在某些对于电路尺寸要求高的场合,小型化也是较为主要的设计指标。

\section{课题的价值及意义}

毫米波通讯和精确制导武器等对Ka波段(26.5$\sim$40 GHz)收发组件以及相关的 MMIC产品都提出了较高的要求。在军用和民用方面对于器件的各种指标包括工作频段和带宽,噪声系数,增益及增益平坦度,稳定性,端口驻波比等都有了更高的要求。随着MMIC工艺的发展,毫米波电路逐渐向单片集成化方向发展,很多毫米波电路都集成在很小尺寸的GaAs集成电路上。小型化、可靠性高、成本低是毫米波发展的趋势\citeup{3}。
低噪声放大器(LNA)一般处于接收系统的前端,其作用是放大从天线接收下来的信号,与此同时还要保证对后级电路的信噪比影响尽可能的小,以确保整个系统正常的工作。而Ka波段MMIC低噪声放大器 (LNA)作为毫米波收发组件接收前端的最重要器件,它的性能对整个接收系统的噪声和灵敏度都有很大的影响,所以设计一个性能优良的Ka波段低噪声放大器(LNA)对于设计整个毫米波收发系统起到非常关键的作用。

\section{低噪声放大器国内外研究现状}

对于低噪声放大器的研究起步较晚,近些年来发展的尤为迅速,但是受器件和工艺条件等因素限制,跟国外的水平仍然是有一定的差距。

2001年邹泉涌团队设计出了一款工作在Ka波段的低噪声放大器,安装在介电常数为2.2,板材厚度0.254mm的RD5880基板上,由两级级联的HEMT低噪声放大芯片组成。在整个设计的过程中,第一级按照最小噪声系数匹配,第二极按照最大传输功率匹配。而且,在两级低噪声放大芯片之间使用了隔板对两级芯片进行隔离各屏蔽,以防止因为两级级间耦合带来的自激现象。经过测量,该放大器在32$\sim$36GHz频带内噪声系数$\leqslant$3dB,增益$\geqslant$25dB\citeup{4}。

2007年中国科学院上海微系统与信息技术研究所的王闯等人在0.25微米 pHEMT 工艺技术的基础上设计了一种毫米波MMIC低噪声放大器。整个放大器由三级级联的电路组成,第一级电路采用源级负反馈的技术来获得较小的噪声以及较低的驻波比,还设计了偏置电路用来隔离射频信号和直流信号之间相互的影响,并且采取了单电源供电的方式。测试结果为在26$\sim$38GHz频带内低噪声放大器噪声系数小于3.0dB,增益大于15dB,端口驻波小于2.1dB压缩点输出功率大于15dBm,芯片尺寸是1mm$\times$2mm$\times$0.1mm,是当时国内报导的面积最小、性能最好的毫米波低噪声放大器\citeup{5}。

国外目前在低噪声方面也已经有了不少成果,所采用的技术手段也日趋成熟,与此同时,在半导体掺杂方面也在不断地改进,制造出了各种性能十分优异的低噪声放大器,各类文献报道也比较多。

2000年 Colin S.Whelan 实现了一种采用含60\%铟的变异 HEMT (Metamorphic\linebreak[4] HEMT-MHEMT)技术实现的低噪声放大器,在26GHz的噪声系数有0.61dB,相对应的资用功率增益为11.8dB,在30$\sim$32GHz范围内二级低噪声放大器的噪声系数为\linebreak[3]1.25$\sim$1.4dB,相对应的资用功率增益为15$\sim$16.5dB,三级低噪声放大器的噪声为1.4$\sim$1.7dB,资用功率增益为24.2$\sim$26.3dB\citeup{6}。

2004年,Jonathan B.Hacker,Joshua Bergman等人研发了一种采用 InAs/AISb HEMT技术制成的三级LNA,在GaAs衬底上,在34$\sim$36GHz的工作频率范围内,噪声系数仅为2.1dB,相关的增益为22dB\citeup{7}。

对比以上国内国外的研究状况,我国近些年来低噪声放大器的发展势头不错,但是由于起步较晚,和一些起步较早的国家仍有一些差距。其中最主要的原因有两个方面,一方面是微波集成电路的加工环节方面,一些基础生产材料,测量仪器和制造设备的仍然受制于人;另一个方面是国内的MMIC的生产线还比较少,一些工艺方面的如成品率,模型的精确度等还需要进一步改进。但是由于Ka波段的低噪声放大器在军事方面也有一定的应用,所以在这个方面技术比较先进的国家也长期对我国实行技术封锁,所以按照目前的形势来看,十分迫切的需要依靠我国自身的力量,去提高芯片的加工工艺和设计能力。

\section{本文主要研究内容}

本文主要研究内容为对于Ka波段的低噪声放大器的设计,并仿真研究该放大器的增益,带宽特性。设计指标如下:
\begin{itemize}
	\item 工作频率:30$\sim$42GHz
	\item 噪声系数:$<$2.8dB
	\item 增益:$>$22dB
\end{itemize}

由于研发阶段比较高昂的流片费用,本次设计在满足以上所有指标的基础上,争取做到尽量减少元器件,使得芯片的面积达到尽可能小,从而控制成本。