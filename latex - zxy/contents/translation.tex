% !Mode:: "TeX:UTF-8"

\chapter{外文资料翻译}

\section{题目及摘要}

\paragraph{题目} 微波功率模块:高功率传送器的通用射频模块

\paragraph{摘要} 描述一个微波功率模块,一个微型的射频功率放大器结合一个低噪声高增益单片微波集成电路固态放大器由一个高效率低增益真空功率升压螺旋行波管和和一个集成功率调节器驱动。100-200W的功率升压行波管和2-40GHz的微波功率模块的进步和发展,是关于射频性能的很好的例子。比较微波功率模块的优点和传统的TWTA和SSPA方法,和讨论各种基于微波功率模块的发射机结构。各种微波功率模块发射机的应用在雷达,电子对抗,通信被期待在不久的将来得到广泛的应用。

关键词:电子管,高功率射频接收机,微波功率放大器,微波功率模块,微波管,毫米波管,行波管,真空电子设备。

\section{介绍}

微波功率模块是高度微型的和高度集成的放大器操作在微波和毫米波频段。如图一,三个不同的元件,一个行波管,集成功率调节器和固态放大器,组合成一个紧密封装的功率模块“超级组件”。图片二是 Northrop Grumman 的超带块微波功率模块的顶端和底端。在微波功率模块中,高功率,高效率和宽频带特性的真空电子和一个低噪声,信号处理技术相结合。这个补充的技术对真空电子和固态是有利的,可以通过分割等量的在 SSA 和 TWT 之间的增益来获得。出于这个原因微波功率模块行波管通常指的是真空功率升压行波管。微波功率模块的主要功率是依靠 MIL-STD-704E 标准化的,定义了被允许的稳定的状态和短暂的变化(220-320V)到标准的 270V 直流输入电压。通过使用额外的外部电压转换器,其他主要功率选项可以很容易的获得。

按照设计需求,功率模块典型的是从50-200W的连续波射频输出功率带有大于50dB的增益超过带宽两个八度。需要脉冲,例如对于雷达的,很容易处理通过合并高速电子调制器提供短暂上升和下降时间在10ns以下在到达高达330KHz脉冲重复频率。微波功率模块已经展示出了很杰出的效率特性,达到50\%,在理想的窄带宽结构下操作,并且大于35\%在多阶带宽设计下。在射频链中,通过合适的增益分配,预先不可获得的低噪声性能(噪声系数小于10dB)对于一个高增益,高功率放大器已经被实现。微波功率放大模块提供很重要的性能上的提升超过了固态功率放大器和传统的行波管放大器对于窄带宽和宽频带的放大器在大于2GHz下操作。这个已经被普遍认可在三个R\&D,100多个奖项中,对于6-18GHz微波功率模块,C-Band微波功率模块和超宽带微波功率模块,各自被公认为全世界100项最重要科技产品之一,分别在1994,1997和1998年。

微波功率模块对于高性能接收机结构是非常通用的构件爱你,适当的功率需要有限的空间,宽度和和冷却的规定,一个单独的微波功率模块可以被作为一个自动控制的微型发射机。应用需要更高的功率可简单的实现通过功率多单元整合,提供额外的优点对于获得电波控制和高效率的辐射功率,微波功率模块可以被用到一个相控多元阵中。

这篇论文介绍了微波功率模块技术,包括设计的细节,性能和系统的应用。这篇文章技术的说明采用了微波功率模块在Northrop Grumman防御系统分支发展的例子来说明。微波功率模块展现了一个新的技术在微波和毫米波功率放大器上,并且只有一个有限的科技信息在参考文献中。读者感兴趣的其余的信息关于微波功率模块可以在网站查询产品资料手册和注意事项和更新的动态。

这篇文章的介绍顺序如下。第二部分包括一个简短的介绍关于最初微波功率模块的发展在三军使用的项目下。这个会继续说明在第三和第四部分中通过讨论Northrop Grumman为了真空升压行波管和微波功率模块发展利用性能资料提供宽带和窄带的微波设计和从毫米波模型所得到的初步结果所采用的方法。权衡比较SSPA,行波管放大器和微波功率模块在第七部分展示出的系统设计,然后在第八部分简短的讨论一下可靠性。基于微波功率模块的发射机概念结构在第IX部分进行描述,包括一般的应用。最后,对于目前的地位和未来的预测和发展在第X部分进行总结。

\section{背景}

微波功率模块在1991年后期开始发展作为美国国防部高级研究计划局自发的真空电子的一部分。五个工业发展团队参加了项目:EDD,美国洛克希德马丁公司,Northrop,和Westinghouse以及Varian。