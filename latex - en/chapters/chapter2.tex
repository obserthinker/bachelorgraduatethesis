% !Mode:: "TeX:UTF-8"

\chapter{The Theory of FDTD}
\section{Yee Cell}
Maxwell's equations are a set of equations which can be written in differential form or integral form. They are the foundation of macroscopic electromagnetic phenomenas. There are two kinds of numerical solver to Maxwell's equation. One kind of solvers were developed from integral form of Maxwell's equations, called integral equation solvers, including MoM, BEM etc. Another kind of solvers, like FDTD and FEM, were developed from differential form of Maxwell's equations, called differential equation solvers.

FDTD is based on Maxwell's equations in differential form, which are shown as follows. Then Maxwell's equations are modified by discretized in central-difference way.
\begin{equation}\label{maxwellH}
\nabla\times\mathbf{\mathit{H}}=\frac{\partial \mathbb{\mathit{D}}}{\partial t}+\mathbf{\mathit{J}},
\end{equation}
\begin{equation}\label{maxwellE}
\nabla\times\mathbf{\mathit{E}}=-\frac{\partial \mathbb{\mathit{B}}}{\partial t}-\mathbf{\mathit{J}}_m.
\end{equation}

%各向同性线性介质中的本构关系为
%\begin{equation}\label{bengouguanxi}
%\begin{cases}
%\mathbf{\mathit{D}}&=\varepsilon\mathbf{\mathit{E}}\\
%\mathbf{\mathit{B}}&=\mu\mathbf{\mathit{H}}\\
%\mathbf{\mathit{J}}&=\sigma\mathbf{\mathit{E}}\\
%\mathbf{\mathit{J}}_m&=\sigma_m \mathbf{\mathit{H}}
%\end{cases}.
%\end{equation}
%真空中$$\sigma=0$$,$$\sigma_m=0$$,以及
%\begin{eqnarray*}
%\varepsilon=\varepsilon_0=8.85\times 10^{-12}F/m\\
%\mu=\mu_0=4\pi\times 10^{-7}H/m.
%\end{eqnarray*}

In  Cartesian coordinate system,the equation\eqref{maxwellH} and \eqref{maxwellH} are written in following forms:
\begin{equation}\label{3dmaxwellE}
\begin{cases}
\frac{\partial H_z}{\partial y}-\frac{\partial H_y}{\partial z}=\varepsilon\frac{\partial E_x}{\partial t}+\sigma E_x\\
\frac{\partial H_x}{\partial z}-\frac{\partial H_z}{\partial x}=\varepsilon\frac{\partial E_y}{\partial t}+\sigma E_y\\
\frac{\partial H_y}{\partial x}-\frac{\partial H_x}{\partial y}=\varepsilon\frac{\partial E_z}{\partial t}+\sigma E_z
\end{cases}
\end{equation}

\begin{equation}\label{3dmaxwellH}
\begin{cases}
\frac{\partial E_z}{\partial y}-\frac{\partial H_y}{\partial z}=-\mu\frac{\partial H_x}{\partial t}-\sigma_m H_x\\
\frac{\partial E_x}{\partial z}-\frac{\partial H_z}{\partial x}=-\mu\frac{\partial H_y}{\partial t}-\sigma_m H_y\\
\frac{\partial E_y}{\partial x}-\frac{\partial H_x}{\partial y}=-\mu\frac{\partial H_z}{\partial t}-\sigma_m H_z
\end{cases}.
\end{equation}

上面的两组六个方程式各场分量关于时间、空间的一阶偏导的方程组,并且各个场量相互嵌套。首先要对连续变量进行离散,包含时间离散和空间离散。令 $u(x,y,z,t)$ 代表 $\mathbf{\mathit{E}}$ 或 $\mathbf{\mathit{H}}$ 在直角坐标系中某一分量。在空间方面,我们假设在各方向上空间是均匀离散的,在 $x$、 $y$、 $z$ 方向上离散的网格长度分别为 $\Delta x$、 $\Delta y$、 $\Delta z$,用 $i$、 $j$、 $k$ 分别表示 $x$、 $y$、 $z$ 方向上的网格标识。在时间方面,我们假设在时间轴上就行均匀离散,离散的时间步长为 $\Delta t$,用字符 $n$ 来表示时间 $t$ 的时刻标识。这样,某一场分量咋看时间和空间域中的离散就可以用如下符号表示:
\begin{equation}
u(x,y,z,t)=u(i\Delta x,j\Delta y,k\Delta z,n\Delta t)=u^n(i,j,k).
\end{equation}

一阶偏导数的有限差分格式有多种,例如前向差分、后向差分、中心差分以及指数差分。这里我们选择具有二阶精度的中心差分法。所以,以 $x$ 方向为例,场分量对空间一阶偏导的差分格式为
\begin{equation}\label{space discrete}
\frac{\partial u^n(i,j,k)}{\partial x} \approx \frac{
	u^n(i+\frac{1}{2},j,k)-u^n(i-\frac{1}{2},j,k)
	}{\Delta x},
\end{equation}
场分量对时间的一阶偏导的差分格式为
\begin{equation}\label{time discrete}
\frac{\partial u^n(i,j,k)}{\partial t} \approx \frac{
	u^{n+\frac{1}{2}}(i,j,k)-u^{n-\frac{1}{2}}(i,j,k)
}{\Delta x}.
\end{equation}
现在对\eqref{3dmaxwellH}和\eqref{3dmaxwellE}中的六个方程都进行了离散。接下来就要考虑在空间和时间中对这些离散节点的排布。

在空间中,FDTD 离散的电场和磁场各节点的空间排布如图\ref{fig: yee cell}所示。这就是 Yee 提出的 Yee 元胞的空间离散结构。

\begin{figure}[hp]
\centering
		\begin{tikzpicture}
		\def \len {6}
		\def \hlen {3}
		\def \coe {-0.75}
		
		%back rectangle wigh dashline
		\draw(0,0) rectangle +(\len,\len);
		\draw [dashed] ($(0,0)+0.5*(0,\len)$) -- +(\len,0);
		\draw [dashed] ($(0,0)+0.5*(\len,0)$) -- +(0,\len);
		
		%front rectangle wigh dashline
		\draw($(0,0)+\coe*(\hlen,\hlen)$) rectangle +(\len,\len);
		\draw [dashed] ($(0,0)+(0,\hlen)+\coe*(\hlen,\hlen)$) -- +(\len,0);
		\draw [dashed] ($(0,0)+(\hlen,0)+\coe*(\hlen,\hlen)$) -- +(0,\len);
		
		%connect two rectangle
		\draw (0,0) -- ($(0,0)+\coe*(\hlen,\hlen)$);
		\draw ($(0,0)+(0,\len)$) -- ($(0,0)+(0,\len)+\coe*(\hlen,\hlen)$);
		\draw ($(0,0)+(\len,0)$) -- ($(0,0)+(\len,0)+\coe*(\hlen,\hlen)$);
		\draw ($(0,0)+(\len,\len)$) -- ($(0,0)+(\len,\len)+\coe*(\hlen,\hlen)$);
		
		%all dashline
		\draw [dashed] ($(0,0)+(0,\hlen)$) -- ($(0,0)+(0,\hlen)+\coe*(\hlen,\hlen)$);
		\draw [dashed] ($(0,0)+(\hlen,0)$) -- ($(0,0)+(\hlen,0)+\coe*(\hlen,\hlen)$);
		\draw [dashed] ($(0,0)+(\len,\hlen)$) -- ($(0,0)+(\len,\hlen)+\coe*(\hlen,\hlen)$);
		\draw [dashed] ($(0,0)+(\hlen,\len)$) -- ($(0,0)+(\hlen,\len)+\coe*(\hlen,\hlen)$);
		
		\draw [dashed] ($(0,0)+0.5*\coe*(\hlen,\hlen)$) -- ++(\len,0) -- ++(0,\len) -- ++(-\len,0) -- cycle;
		
		%axis
		\draw [->] (0,0) -- +(\len+2,0) node[right]{$y$};
		\draw [->] (0,0) -- +(0,\len+2) node[above]{$z$};
		\draw [->] (0,0) -- +($(0,0)-(\hlen,\hlen)$) node[below,left]{$x$};
		
		%nodes
		%Hx
		\node[shape=circle,fill=cyan,inner sep=0pt] at (\hlen,\hlen) {$H_x$};
		\node[shape=circle,fill=cyan,inner sep=0pt] at ($(\hlen,\hlen)+\coe*(\hlen,\hlen)$) {$H_x$};
		%Hz		
		\node[shape=circle,fill=cyan,inner sep=0pt] at ($(0,0)+(\hlen,0)+0.5*\coe*(\hlen,\hlen)$) {$H_z$};
		\node[shape=circle,fill=cyan,inner sep=0pt] at ($(0,0)+(\hlen,0)+0.5*\coe*(\hlen,\hlen)+(0,\len)$) {$H_z$};
		%Hy
		\node[shape=circle,fill=cyan,inner sep=0pt] at ($(0,0)+(0,\hlen)+0.5*\coe*(\hlen,\hlen)$) {$H_y$};
		\node[shape=circle,fill=cyan,inner sep=0pt] at ($(0,0)+(0,\hlen)+0.5*\coe*(\hlen,\hlen)+(\len,0)$) {$H_y$};
		%Ez
		\node[shape=circle,fill=pink,inner sep=0pt] at ($(0,\hlen)$) {$E_z$};
		\node[shape=circle,fill=pink,inner sep=0pt] at ($(0,\hlen)+\coe*(\hlen,\hlen)$) {$E_z$};
		\node[shape=circle,fill=pink,inner sep=0pt] at ($(0,\hlen)+(\len,0)$) {$E_z$};
		\node[shape=circle,fill=pink,inner sep=0pt] at ($(0,\hlen)+\coe*(\hlen,\hlen)+(\len,0)$) {$E_z$};
		%Ex
		\node[shape=circle,fill=pink,inner sep=0pt] at ($(0,\hlen)+0.5*\coe*(\hlen,\hlen)+(0,\hlen)$) {$E_x$};
		\node[shape=circle,fill=pink,inner sep=0pt] at ($(0,\hlen)+0.5*\coe*(\hlen,\hlen)+(\len,0)+(0,\hlen)$) {$E_x$};
		\node[shape=circle,fill=pink,inner sep=0pt] at ($(0,\hlen)+0.5*\coe*(\hlen,\hlen)+(0,-\hlen)$) {$E_x$};
		\node[shape=circle,fill=pink,inner sep=0pt] at ($(0,\hlen)+0.5*\coe*(\hlen,\hlen)+(\len,0)+(0,-\hlen)$) {$E_x$};
		%Ey
		\node[shape=circle,fill=pink,inner sep=0pt] at ($(\hlen,\hlen)+(0,\hlen)$) {$E_y$};
		\node[shape=circle,fill=pink,inner sep=0pt] at ($(\hlen,\hlen)+\coe*(\hlen,\hlen)+(0,\hlen)$) {$E_y$};
		\node[shape=circle,fill=pink,inner sep=0pt] at ($(\hlen,\hlen)+(0,-\hlen)$) {$E_y$};
		\node[shape=circle,fill=pink,inner sep=0pt] at ($(\hlen,\hlen)+\coe*(\hlen,\hlen)+(0,-\hlen)$) {$E_y$};
		\end{tikzpicture}
\caption{Yee元胞的空间离散结构}\label{fig: yee cell}
\end{figure}

由图可见,在 Yee 元胞中,$\mathbf{\mathit{E}}$和$\mathbf{\mathit{H}}$的各自三个场分量在 Yee 元胞表面上进行离散。电场分量$\mathit{E}_x$、$\mathit{E}_y$和$\mathit{E}_z$在 Yee 元胞的棱边中间,方向与棱边一致,磁场分量$\mathit{E}_x$、$\mathit{E}_y$和$\mathit{E}_z$在 Yee 元胞的表面的中心,其方向垂直元胞面。每一个电场分量周围都有四个电场分量环绕,同样,每个磁场分量周围由四个电场分量环绕。这种取样方式符合法拉第感应定律和安培环路定律的自然结构。同时这种结构也适合麦克斯韦方程的差分计算,能够恰当的描述电磁场的传播特性。

在时间上,电场和磁场相差半个时间步长,交替离散,从而可以在时间上迭代求解,而不需要进行矩阵求逆运算。

综合空间和时间离散排布结构来看,Yee 元胞可以自然的描述电磁之间相互作用的过程,从而给定相应电磁问题的初始值以及边界条件,就可以利用 FDTD 方法逐时间步推进的求得以后各个时刻空间电磁场的分布。

为了准确描述每一个场分量离散点的位置,我们需要对空间位置进行编号。根据上述各个场分量的空间排布,定义各个方向上的 Yee 元胞棱边为整数编号,棱边的中间位置为半整数编号。具体各个分量的取值如下表所述。

\begin{table}
\caption{Yee元胞中$\mathbf{\mathit{E}}$和$\mathbf{\mathit{H}}$各分量节点的位置}
	\centering
	\begin{tabular}{ccccc}
		\toprule
		\multirow{2}{3em}{场分量} & \multicolumn{3}{c}{空间分布位置} & \multirow{2}{6em}{时间分布位置}\\
		\cline{2-4}
		& $x$ & $y$ & $z$ &\\
		
		\midrule
		$E_x$ & $i+\frac{1}{2}$ & $j$ & $k$ & \multirow{3}{1em}{$n$}\\
		$E_y$ & $i$ & $j+\frac{1}{2}$ & $k$ & \\
		$E_z$ & $i$ & $j$ & $k+\frac{1}{2}$ & \\
		
		$H_x$ & $i$ & $j+\frac{1}{2}$ & $k+\frac{1}{2}$ & \multirow{3}{3em}{$n+\frac{1}{2}$}\\
		$H_y$ & $i+\frac{1}{2}$ & $j$ & $k+\frac{1}{2}$ & \\
		$H_z$ & $i+\frac{1}{2}$ & $j+\frac{1}{2}$ & $k$ & \\		
		\bottomrule
	\end{tabular}
\end{table}

\section{Yee元胞的Maxwell离散形式}

我们以\eqref{3dmaxwellE}的第一式为例来说明 Yee 元胞的 Maxwell 方程差分格式。根据\eqref{space discrete}和\linebreak[1]\eqref{time discrete}的差分格式,对\eqref{3dmaxwellH}中第一式的各个部分写成离散格式,得到下式。

\begin{equation}\label{ex discrete}
\begin{split}
E^{n+1}_{x}\left( i+\frac{1}{2},j,k \right)=&CA(m) \cdot E^{n}_{x}\left( i+\frac{1}{2},j,k \right)+CB(m)\\
{}&\cdot\left[
\frac{H^{n+\frac{1}{2}}_{z}\left(i+\frac{1}{2},j+\frac{1}{2},k\right)-H^{n+\frac{1}{2}}_{z}\left(i+\frac{1}{2},j-\frac{1}{2},k\right)}{\Delta y}\right.\\
{}&-
\left.\frac{H^{n+\frac{1}{2}}_{y}\left(i+\frac{1}{2},j,k+\frac{1}{2}\right)-H^{n+\frac{1}{2}}_{z}\left(i+\frac{1}{2},j,k-\frac{1}{2}\right)}{\Delta z}
\right].
\end{split}
\end{equation}
其中
\begin{equation}
CA(m)=\frac{
	1-\frac{\sigma(m)\Delta t}{2\varepsilon(m)}
	}{
	1-\frac{\sigma(m)\Delta t}{2\varepsilon(m)}	
	}
\end{equation}

\begin{equation}
CB(m)=\frac{
	\frac{\Delta t}{\varepsilon(m)}
}{
1+\frac{\sigma(m)\Delta t}{2\varepsilon(m)}	
}	
\end{equation}\label{ex discrete simple}
$m$为等号左端场分量的坐标。本文中,为能更简洁说明方法起见,我们假设介质全为无耗介质,即$\sigma=\sigma(m)=0$,则有$CA(m)=1$,$CB(m)=\frac{\Delta t}{\varepsilon(m)}$,所以式\eqref{ex discrete}变为
\begin{equation}
\begin{split}
E^{n+1}_{x}\left( i+\frac{1}{2},j,k \right)=&E^{n}_{x}\left( i+\frac{1}{2},j,k \right)+\frac{\Delta t}{\varepsilon(m)}\\
{}&\cdot\left[
\frac{H^{n+\frac{1}{2}}_{z}\left(i+\frac{1}{2},j+\frac{1}{2},k\right)-H^{n+\frac{1}{2}}_{z}\left(i+\frac{1}{2},j-\frac{1}{2},k\right)}{\Delta y}\right.\\
{}&-
\left.\frac{H^{n+\frac{1}{2}}_{y}\left(i+\frac{1}{2},j,k+\frac{1}{2}\right)-H^{n+\frac{1}{2}}_{z}\left(i+\frac{1}{2},j,k-\frac{1}{2}\right)}{\Delta z}
\right].
\end{split}
\end{equation}

按照同样的方法对$\mathbf{\mathit{H}}$的场分量进行整理。以$H_x$为例,式\eqref{3dmaxwellH}的第一式整理之后得到
\begin{equation}\label{hx discrete simple}
\begin{split}
H^{n+\frac{1}{2}}_{x}\left( i,j+\frac{1}{2},k+\frac{1}{2} \right)=&H^{n-\frac{1}{2}}_{x}\left( i,j+\frac{1}{2},k+\frac{1}{2} \right)+\frac{\Delta t}{\mu}\\
{}&\cdot\left[
\frac{
	E^n_y\left(i,j+\frac{1}{2},k+1\right)-E^n_y\left(i,j+\frac{1}{2},k\right)
	}{\Delta z}\right.\\
{}&+\left.
\frac{
	E^n_z\left(i,j,k+\frac{1}{2}\right)-E^n_z\left(i,j+1,k+\frac{1}{2}\right)
}{\Delta z}
\right]
\end{split}
\end{equation}
其余的各个$\mathbf{\mathit{E}}$和$\mathbf{\mathit{H}}$的场分量的离散格式按照上述两式的方法整理,于是我们得到各个场分量关于时间迭代的离散格式。

\section{Courant稳定性条件}
根据上一节的内容,我们知道 FDTD 是用求解一组离散的有限差分方程来代替求解连续的偏微分方程组,即麦克斯韦旋度方程。所以和和许多数值方法相同,需要保证离散后的差分方程组的解释收敛和稳定的,否则是没有意义的。

首先考虑对时间间隔$\Delta t$的限制。由于任意场量在时域上可以分解为时谐场的叠加,所以考察对于时谐场的情况:
\begin{equation}\label{stability: shi xie chang}
u(x,y,z,t)=u_0exp(j\omega t)
\end{equation}
式\eqref{shi xie chang}对时间的一阶偏微分为:
\begin{equation}\label{stability: u partial t}
\frac{\partial u}{\partial t}=j\omega u
\end{equation}
式\eqref{stability: u partial t}的左端用中心差分离散近似代替得到:
\begin{equation}\label{stability: chafen}
\frac{u^{n+\frac{1}{2}}-u^{n-\frac{1}{2}}}{\Delta t}=j\omega u^n
\end{equation}
其中$u_n=u(x,y,z,n\Delta t)$。定义增长因子$q$为
\begin{equation}\label{stability: q}
	q=\frac{u^{n+\frac{1}{2}}}{u^n}=\frac{u^n}{u^{n-\frac{1}{2}}}
\end{equation}
将式\eqref{stability: q}带入\eqref{stability: chafen}得到方程
\begin{equation}
q^2-j\omega \Delta tq-1=0
\end{equation}
解方程,得到
\begin{equation}
q=\frac{j\omega \Delta t}{2}\pm\sqrt{1-\left(\frac{\omega \Delta  t}{2}\right)^2}
\end{equation}
如果要让随着时间步进的场量不发散,则需要让$|q|\leqslant 1$,由此得到
\begin{equation}\label{stability: dt}
\frac{\omega \Delta t}{2}\leqslant 1
\end{equation}
这就是一般场量对时间间隔$\Delta t$的稳定性要求。

面对包含6个场分量的Maxwell方程组,可以导出电磁场任意直角分量均满足齐次波动方程
\begin{equation}\label{stability: qicibodong}
\frac{\partial^2 f}{\partial x^2}+\frac{\partial^2 f}{\partial y^2}+\frac{\partial^2 f}{\partial z^2}+\frac{\omega^2}{c^2}f=0
\end{equation}
由于任意波都可以展开为平面波谱,所以考虑平面波的解:
\begin{equation}\label{stability: pingmianbo}
u(x,y,z,t)=u_0 exp[-j(k_x x+k_y y+k_z z-\omega t)]
\end{equation}
将\eqref{stability: pingmianbo}带入\eqref{stability: qicibodong}并采用有限差分离散,得到
\begin{equation}\label{stability: qicibodong lisan}
\frac{\sin^2 \left( \frac{k_x\Delta x}{2} \right)}{\left(\frac{\Delta x}{2}\right)^2}+
\frac{\sin^2 \left( \frac{k_y\Delta y}{2} \right)}{\left(\frac{\Delta y}{2}\right)^2}+
\frac{\sin^2 \left( \frac{k_z\Delta z}{2} \right)}{\left(\frac{\Delta z}{2}\right)^2}-
\frac{\omega^2}{c^2}=0
\end{equation}
其中$c$为介质中的光速。该式整理并带入\eqref{stability: dt}后得到
\begin{equation}
\left(\frac{c\Delta t}{2}\right)^2
\left[
\frac{\sin^2 \left( \frac{k_x\Delta x}{2} \right)}{\left(\frac{\Delta x}{2}\right)^2}+
\frac{\sin^2 \left( \frac{k_y\Delta y}{2} \right)}{\left(\frac{\Delta y}{2}\right)^2}+
\frac{\sin^2 \left( \frac{k_z\Delta z}{2} \right)}{\left(\frac{\Delta z}{2}\right)^2}
\right]=
\left(\frac{\omega\Delta t}{2}\right)^2\leqslant 1
\end{equation}
该式的成立条件是
\begin{equation}\label{stability: courant}
c\Delta t\leqslant
\frac{1}{
	\sqrt{\frac{1}{(\Delta x)^2}\frac{1}{(\Delta y)^2}+\frac{1}{(\Delta z)^2}}
	}
\end{equation}
式\eqref{stability: courant}给出了空间和时间离散间隔之间应当满足的关系,称为 Courant 稳定性条件。

\section{空间间隔的要求}
在上一节中,时间间隔和空间间隔之间的关系得到了确定,接下来还需要确定空间间隔。以一维的情形为例。由\eqref{stability: qicibodong lisan}得到
\begin{equation}
\frac{\sin^2 \left( \frac{k_x\Delta x}{2} \right)}{\left(\frac{\Delta x}{2}\right)^2}-
\frac{\omega^2}{c^2}=0
\end{equation}
我们看到,只有当$\Delta x\rightarrow 0$时,才能不出现色散,所以我们需要估计$\Delta x$小到什么程度可以接近0.根据三角函数,当$\phi\leqslant \pi/12$时,$sin\phi\approx\phi$,于是得到
\begin{equation}\label{stability: space}
	\Delta x\leqslant \frac{2\pi}{12k}=\frac{\lambda}{12}
\end{equation}
这就是对空间离散间隔的要求。对于多维情形,各个维度的空间间隔同样满足关系式\eqref{stability: space}即可。对于非时谐的宽带时域信号,只要空间离散间隔满足信号的最小波长可以满足关系式\eqref{stability: space}即可。


\section{边界条件}
FDTD 方法把空间离散成一个个 Yee 元胞,然而实际上仿真的时候由于计算机资源的限制,不可能将无限空间仿真出来。另一方面,我们也只需要特定区域的电磁信息。基于这两点,我们需要在仿真空间边界处设定边界条件,以便在只计算仿真空间电磁信息的条件下得到更精确的实际情形中该区域的电磁信息。边界条件可以分为截断边界条件和吸收边界条件。在实际中应用的较为广泛的是吸收边界条件,其中又以 Mur 吸收边界条件和 PML 吸收边界条件为主。


%

%
%\subsection{PML边界条件} PML也是一种吸收边界条件。该吸收边界条件的思路是在FDTD仿真区域外部设置一种特殊的介质层,该介质层和仿真区域有完全匹配的波阻抗,因此入射波将无反射的穿过分界面进入PML层。又由于PML为有耗介质,电磁波进入PML之后将迅速衰减,因此可以只设置有限层数的PML。这里以一种坐标伸缩的PML(Convolution PML,CPML)为例来说明。

\section{并行计算}

由于 Courant 稳定性条件,仿真计算区域的 Yee 网格必须分的足够细,这就限制 FDTD 在大型电磁计算问题中的应用。针对这一问题,并行计算是一个十分有效的解决方案。

计算机领域的并行计算一般分为三层。最上层是任务并行,即把需要计算机处理的任务切割为若干个子任务,将子任务分配给一些计算核心同时计算。在计算核心完成各自的计算之后,采用某种方法将各个子任务的计算结果组合起来得到完整的计算结果。由于FDTD的特点,即各个子区域在每个时间间隔中只需和相邻的子区域交换交界处的数据,使得FDTD非常适合任务并行。针对这个层次的并行,目前有基于 MPI(Massage Passing Interface)和 PVM(Parallel Virtual Machine)这两种消息传递编程的方案。

并行的中层是指令并行,这个层次的并行大多数是由 CPU 制造厂商在开发 CPU 时处理的,FDTD 的并行计算基本不涉及这个层次的并行。

最底层的并行是数据并行。针对一些计算任务,在经过合适的规划后,可以在 CPU 执行一条指令时同时处理多个数据。在 FDTD 中的计算中,由于每个场分量的所有节点都有相同的计算公式,所以经过合适的安排每个场分量的数组,可以在每次计算时计算相邻的多个场点。针对这个层次的并行,也有了一些利用矢量处理器的加速方法。

\section{本章总结}

在这一章中,我们介绍了 FDTD 算法的原理以及 FDTD 离散形式,FDTD 方法的稳定性和收敛性和满足稳定、收敛的条件,FDTD 在仿真开区域电磁场时需要采用的边界条件,以及使用并行计算加速 FDTD 的方法。