% !Mode:: "TeX:UTF-8"

\begin{Cabstract}{时域有限差分}{CUDA}{矢量处理器}{数据并行}{硬件加速}
本文针对时域有限差分(Finite Difference Time Domain,FDTD)方法的两种主流硬件并行加速途径进行了研究。针对使用矢量处理器(Vector Algorithm Logic Unit,VALU)的加速方法,首先进行了加速的理论分析,然后基于现有方案,同时结合FDTD的边界条件的特征,提出了一种新的使用VALU加速计算的计算模型。经过实验,相比以往方案改计算模型可以减少约3.45\%的计算时间。针对使用统一计算设备架构(Compute Unified Device Architecture,CUDA)的图像处理器(Graphic Processor Unit,GPU)的加速方法,我们实现了对截断边界、Mur吸收边界以及CPML吸收边界的加速。对比两种硬件并行加速方案,我们可以看到CUDA有更优越的表现。
\end{Cabstract}
