% !Mode:: "TeX:UTF-8"

\chapter{结束语}

本文针对 FDTD 算法的加速问题,探讨了使用 CPU 内置矢量处理器对其加速的方法,以及使用独立设备 GPU 对其加速的方法。针对两种方法,都给出了具体的实现方案,并以二维 TM 波的仿真为例进行了测试,对测试结果进行了分析与说明,得出了各个加速方案自身的加速效果。针对使用矢量处理器加速的情形,本文提出一种基于传统数据并行计算的改进方案,通过测试,改进方案比传统数据并行方案更有效。在使用外部设备,即 GPU 的情况下,在 CUDA 平台下的 GPU 对 FDTD 方法加速的效果卓越,因此本文认为,使用 GPU 进行计算是未来的趋势。

通过实例测试我们也看出来本文依然存在一些待解决的问题。其中最主要的是使用 CUDA 平台进行加速计算时对程序更为谨慎和全面的考虑。例如,使用常量内存来存储 FDTD 公式中的不变的参数而不是全局内存,以便减小内存带宽。在使用 CUDA 加速时,我们可以借鉴任务并行的 FDTD 算法,即将仿真区域分割成多个子区域,每个区域分派给一个独立的 CUDA 设备,多个 CUDA 设备协作计算。

在本次课题的研究中,在知识能力方面,我深入了解了 FDTD 算法以及该算法的多种边界条件。学习了 CUDA 平台的框架及其编程模型。在探究能力方面,学会了独立的收集文献,整理文献,在前人的科研成果上做进一步的研究。了解了基本的科研方法和思路。
