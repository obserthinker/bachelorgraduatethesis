% !Mode:: "TeX:UTF-8"

\chapter{Conclusion}

In this paper, we focus on the topic of accelerating FDTD with hardwares. First, we discussed accelerating FDTD with a built-in CPU component, VP. For using VP, the theory and a practical scheme were discussed in details. Furthermore, we modified the traditional data parallelism scheme by removing some unnecessary discrete field points. To test the advantage of our modified scheme, the 2D FDTD in TM mode were taken as an example. After the experience, we analyzed the profiling reports. Then, for executing FDTD with CUDA, we do the same things. After that, the conclusion we obtained is that the modified data parallelism scheme is better than traditional scheme, however, compared to the CUDA, it is still very inefficient. The CUDA is a powerful potential computational power waiting to be utilized in the future.

Nevertheless, there are still some places waiting to be enhanced. For example, using constant memory to store some constants, dividing the simulation area into some sub-areas.

In the research of this topic, I have learned much knowledge about FDTD algorithm, CUDA, and vector processor. Gained some valuable experience of dealing with a project has complex organization and huge size. Besides, learned the rules of doing research, including how to collect previous research contributions, find the point to enhance, examine new ideas etc., which is the solid foundation for future growth.